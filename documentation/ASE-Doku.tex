% ------------------------------------------------------------
% LaTeX Template für die DHBW zum Schnellstart!
% Original: https://github.wdf.sap.corp/vtgermany/LaTeX-Template-DHBW
% ------------------------------------------------------------
% ---- Präambel mit Angaben zum Dokument
\input{Inhalt/00_Latex/praeambel}

% ---- Elektronische Version oder Gedruckte Version?
% ---- Unterschied: Die elektronische Version enthält keinen Platzhalter für die Unterschrift
\usepackage{ifthen}
\newboolean{e-Abgabe}
\setboolean{e-Abgabe}{false}    % false=gedruckte Fassung

% ---- Persönlichen Daten:
\newcommand{\titel}{Advanced Software Engineering}
\newcommand{\titelheader}{ASE-Doku}
\newcommand{\arbeit}{Projektarbeitsdokumentation}
\newcommand{\studiengang}{Informatik}
\newcommand{\studienjahr}{2022}
\newcommand{\autor}{Dominik Ochs}
\newcommand{\autorReverse}{Ochs, Dominik}
\newcommand{\verfassungsort}{Karlsruhe}
\newcommand{\matrikelnr}{2847475}
\newcommand{\kurs}{TINF20B2}
\newcommand{\bearbeitungsmonat}{Mai 2022}
\newcommand{\abgabe}{XX. Mai 2023}
\newcommand{\bearbeitungszeitraum}{01.10.2022 - XX.05.2023}
\newcommand{\firmaName}{SAP SE}
\newcommand{\firmaStrasse}{Dietmar-Hopp-Allee 16}
\newcommand{\firmaPlz}{69190 Walldorf, Deutschland}
\newcommand{\betreuerFirma}{Christoph Eckert}
\newcommand{\betreuerDhbw}{Dr. Lars Briem}

\input{Inhalt/00_Latex/kopfundFusszeile}

% ---- Hilfreiches
\newcommand{\zB}{z.\,B. }   % "z.B." mit kleinem Leeraum dazwischen (ohne wäre nicht korrekt)
\newcommand{\dash}{d.\,h. }

\newcommand{\code}[1]{\texttt{#1}} % Ist einfacher zu schreiben als ständig \texttt und erlaubt
                                   % Änderungen im Nachhinein, wenn man z.B. Inline-Code anders stylen möchte.

% ---- Silbentrennung (falls LaTeX defaults falsch / nicht gewünscht sind)
\hyphenation{HANA}         % anstatt HA-NA
\hyphenation{Graph-Script} % anstatt GraphS-cript
\hyphenation{Performance-tests}

% ---- Watermark/Wasserzeichen
%\input{Inhalt/00_Latex/watermark} % Auskommentieren wenn nicht erwünscht
%\watermark{gray}{\textbf{DRAFT}}     % Auskommentieren wenn nicht erwünscht. Nimmt optional die opacity/Deckraft z.B. \watermark[0.1]{green}{text} für 10% Deckkraft
%\SetWatermarkSize{8} % Optional. Standard ist 5.8. 

% ---- Beginn des Dokuments
\begin{document}
\setlength{\parindent}{0pt}              % Keine Paragraphen Einrückung.
                                         % Dafür haben wir den Abstand zwischen den Paragraphen.
\setcounter{secnumdepth}{2}              % Nummerierungstiefe fürs Inhaltsverzeichnis
\setcounter{tocdepth}{1}                 % Tiefe des Inhaltsverzeichnisses. Ggf. so anpassen,
                                         % dass das Verzeichnis auf eine Seite passt.
%\sffamily                                % Serifenlose Schrift verwenden.

% ---- Vorspann
% ------ Titelseite
\singlespacing
\include{Inhalt/01_Standard/titelseite}  % Titelseite
\newcounter{savepage}
\pagenumbering{Roman}                    % Römische Seitenzahlen
\onehalfspacing

% ------ Erklärung, Sperrvermerk, Abstact
%\include{Inhalt/01_Standard/sperrvermerk}
%\include{Inhalt/01_Standard/erklaerung}
%\include{Inhalt/02_Abstract/abstract-en}
%\include{Inhalt/02_Abstract/abstract-de}

% ------ Inhaltsverzeichnis
\singlespacing
\tableofcontents

% ------ Verzeichnisse
\renewcommand*{\chapterpagestyle}{plain}
\pagestyle{plain}
%\include{Inhalt/03_Verzeichnisse/abkuerzungen}
\listoffigures                          % Erzeugen des Abbildungsverzeichnisses 
%\listoftables                           % Erzeugen des Tabellenverzeichnisses
\renewcommand{\lstlistlistingname}{Quellcodeverzeichnis}
\lstlistoflistings                      % Erzeugen des Listenverzeichnisses
\setcounter{savepage}{\value{page}}


% ---- Inhalt der Arbeit
\cleardoublepage
\pagenumbering{arabic}                  % Arabische Seitenzahlen für den Hauptteil
\setlength{\parskip}{0.5\baselineskip}  % Abstand zwischen Absätzen
\rmfamily
\renewcommand*{\chapterpagestyle}{scrheadings}
\pagestyle{scrheadings}
\onehalfspacing

\chapter{Einführung}

\section{Übersicht über die Applikation}

Bei dieser Applikation handelt es sich um ein singleplayer Kartenspiel. 
Hier versucht ein Spieler auf einer einsamen Insel zu überleben und von ihr zu flüchten, 
indem er Karten zieht und Ressourcen (\textit{wood}, \textit{plastic}, \textit{metal}) sammelt, um Gegenstände zu bauen (\textit{Scavange}), 
die ihn vor seinen Feinden (\textit{Encounter}) schützen oder ihm eine Rettung (\textit{Endeavor}) von der Insel ermöglichen. 
Während des Spiels trifft der Spieler auf Tiere (\textit{spider}, \textit{tiger}, \textit{snake}), 
die ihm gesammelte Ressourcen kosten können, oder Katastrophen, 
die Ressourcen zerstören. Das Spiel endet, wenn der Spieler sich retten konnte oder keine Aktionen mehr möglich sind. 

Das Kartenspiel hat verschiedene Spielphasen (s. \autoref{fig:phases}), in denen der Spieler unter bestimmten Bedingungen Karten ziehen 
und Gegenstände bauen kann (\textit{Scavange}). Wenn ein Gegenstand aus der Kategorie Rettungen (\textit{Endeavor}) gebaut wurde, 
muss der Spieler würfeln, um zu entscheiden, ob die Rettung erfolgreich ist. 
Ein Segelboot und ein Hanggleiter stellen eine Rettung dar, 
wenn beim Würfeln eine bestimmte Augenzahl erzielt wird. 
Wenn der Rettungsversuch fehlschlägt, kann der Spieler weiter Karten ziehen und Gegenstände bauen, 
solange Karten im Stapel vorhanden sind. Das Bauen eines Dampfschiffs und eines Heißluftballons 
ist nur mit einer Feuerstelle möglich und garantiert eine erfolgreiche Rettung. Wenn ein Tier gezogen wird, 
muss der Spieler gegen es kämpfen (\textit{Encounter}) und auch hier bestimmt Würfeln über Sieg oder Niederlage.
Für unterschiedliche Aktionen werden unterschiedliche Würfel (\textit{vier-}, \textit{sechs-}, \textit{acht-seitig}) 
verwendet.
Das Spiel endet, wenn keine weiteren Aktionen mehr möglich sind oder wenn eine garantierte Rettung erfolgt ist. 
Wenn das Spiel endet (\textit{End}), bleibt es in diesem Zustand, bis ein neues Spiel gestartet wird, 
bis das Spiel neu initialisiert wird oder die Anwendung beendet wird. 

\begin{figure}
	\centering
	\includegraphics[width=0.6\textwidth]{Bilder/game-phases.png} 
	\caption{Spielphasen.}
	\label{fig:phases}
\end{figure} 

\section{Wie startet man die Applikation?}

Vorraussetzung ist \textbf{Java 17}\footnote{Check mit \texttt{java -version}.}. \\ 
Wahlweise wird eine gängige Java-IDE\footnote{Hier wurde \textit{Jetbrains' IntelliJ} verwendet.} 
\underline{oder} eine \textit{maven}-Installation benötigt.

\subsubsection{Über Maven:} 

Im Wurzelverzeichnis folgende Maven-Befehle ausführen:

\texttt{mvn compile} \\
\texttt{mvn package} \\ 
\texttt{java -jar target/ase-1.0-SNAPSHOT.jar} \\ 
 

\subsubsection{Über IDE:}

Projekt mit IDE im Wurzelverzeichnis öffnen. 
Zu folgendem Pfad navigieren \\ 
\texttt{./src/main/java/de/dhbw/karlsruhe/ase/plugin/cli} \\
und dort die \textit{main}-Methode der Klasse \textit{Main} ausführen lassen.

\subsubsection{Erste Schritte:}

Nun läuft das Command Line Interface der Applikation und es kann mit \\
\texttt{help} \\
eine Übersicht über die möglichen Befehle (Name, Regex, Beschreibung) aufgerufen werden, 
oder mit \\
\texttt{start?} \\
ein Spiel mit zufälligem Kartenstapel gestartet werden. Daraufhin kann mit \\
\texttt{draw} \\
begonnen werden Karten zu ziehen.

\section{Wie testet man die Applikation?}

\subsubsection{Über Maven:}

Im Wurzelverzeichnis:

\texttt{mvn test}

\subsubsection{Über IDE:} 

Die Tests befinden sich unter \\
\texttt{./src/test/java} \\
und können dort mit einer IDE-Funktion ausgeführt werden. \\ 
Es existieren Blackbox-Integration-Tests und Unit-Tests, 
die aber keine besonderen Anforderungen außer eine \textbf{JUnit5}-Abhängigkeit haben.
\chapter{Clean Architecture}

\section{Was ist Clean Architecture?}

Clean Architecture ist ein allgemeines Konzept zum Design von Softwarearchitekturen, das darauf abzielt, 
die Struktur einer Anwendung so zu gestalten, dass sie unabhängig von ihren Benutzerschnittstellen, 
Datenquellen und anderen äußeren Faktoren bleibt. 
Dies bedeutet, dass die verschiedenen Komponenten 
und Funktionen einer Anwendung in abgekapselten Schichten angeordnet werden (\textit{Onion}-Architektur), 
um sicherzustellen, dass Veränderungen in einer Schicht die tieferliegenden Schichten nicht beeinflussen. 
Außere Schichten haben dabei Abhängigkeiten von tieferen Schichten aber niemals umgekehrt.
Aufrufe von tieferen Schichten an äußere Schichten werden über Abstraktionen (\textit{Dependency Inversion} und 
\textit{Injection}) realisiert. Tiefere Schichten sind dabei langlebiger als Schichten weiter außen. \\
Dies führt zu einer Anwendung, die flexibler und leichter zu entwickeln und zu warten ist, 
da Änderungen an einem Teil der Anwendung die tieferliegenden Schichten nicht berühren.

\section{Analyse der Dependency Rule}

Schichten im Sinne der Dependency Rule sind hierp \texttt{plugin}, \texttt{application}, \texttt{domain} 
und \texttt{abstraction} aufsteigend geordnet nach zunehmender Tiefe in der Onion-Architektur\footnote{Siehe \autoref{sec:clean_arch_layers} für mehr Informationen.}. 
Sie werden durch gleichnamige Packages, die parallel unter \\ 
\texttt{de.dhbw.karlsruhe.ase} \\
zu finden sind, erschöpfend repräsentiert. 
Es sind Abhängigkeiten von äußerden Schichten in tiefere Schichten erlaubt, aber nicht umgekehrt.
Diese Regel wird für die oben genannten Schichten \underline{immer} eingehalten, 
was die folgenden beiden Beispiele illustrieren.  

\subsubsection{(Positiv) Beispiel 1:}

\autoref{fig:dep-camp} zeigt das verlangte UML-Diagramm der Beispielklasse 1 \textit{Camp} der Schicht \textit{domain}, 
welches die Dependency Rule einhält. Wie im Diagramm zu sehen ist hängt Camp ausschließlich von 
Klassen in der selben Schicht (\textit{domain}) ab. 
\textit{Crafting} ist dabei keine Schicht sondern ein \textit{Aggregat}\footnote{Siehe \autoref{sec:ddd}.}. 
Schichten sind ausschließlich die oben genannten. \\ 
Weiter zu sehen ist, dass ausschließlich Klassen in \textit{application} von Camp abhängen, aber keine 
aus höheren Schichten (was in diesem konkreten Fall nur \textit{abstraction} sein könnte).

\begin{figure}[H]
	\centering
	\includegraphics[width=1.\textwidth]{Bilder/Camp_structure.pdf} 
	\caption{UML-Diagramm 1 zum Einhalten der Dependency Rule von \textit{Camp}. }
	\label{fig:dep-camp}
\end{figure} 

\subsubsection{(Positiv) Beispiel 2:}

\autoref{fig:dep-game} zeigt ein weiteres positives Beispiel für die Dependency Rule, 
da diese in dieser Softwarearchitektur nicht verletzt wird. Abgebildet ist die Klasse \textit{Game}, 
welche Lediglich von Klassen aus der eigenen Schicht (\textit{application}) und Klassen aus der tieferen 
Schicht \textit{domain} abhängt. Abhänig von Game sind nur Klassen aus der Schicht \textit{plugin},
bzw. dem konkreten \textit{cli}-Plugin. 

\begin{figure}[H]
	\centering
	\includegraphics[width=1.05\textwidth]{Bilder/Game_structure.pdf} 
	\caption{UML-Diagramm 2 zum Einhalten der Dependency Rule von \textit{Game}. }
	\label{fig:dep-game}
\end{figure} 


\section{Analyse der Schichten} \label{sec:clean_arch:layers}

In dieser Softwarearchitektur gibt es folgende Schichten aufsteigend sortiert nach Tiefe: 
\texttt{plugin}, \texttt{application}, \texttt{domain} und \texttt{abstraction}.
Hierbei stellt \textit{plugin} die äußerste und \textit{abstraction} die tiefste Schicht dar. 

\subsubsection{Schicht Plugin:}

Bei einem Refactoring der Plugin-Schicht ist aufgefallen, dass viele der (Fehler-)Hinweise, 
die aufgrund von falschem Befehlssyntax oder einem falschen Befehl an den User über das CLI ausgegeben werden, 
keine gemeinsame einheitliche Form aufwiesen, da diese direkt als String ausgegeben wurden. 
Um eine einheitliche Fehlermeldung zu gewährleisten wurde der \textit{ErrorBuilder} eingefügt, 
dessen Aufgabe es ist, die Fehler einheitlich zu formatieren und auszugeben. \autoref{fig:layer-ErrorBuilder} 
zeigt das UML-Klassendiagramm dieser Klasse. Sie wird im gesamten CLI-Plugincode verwendet um (Fehler-)Hinweise 
auszugeben. \\
Über verschiedene Konstruktoren kann ein Grund für den (Fehler-)Hinweis und eine mögliche Abhilfeemfehlung 
eingegeben werden.
Mit der \textit{print}-Methode kann der erzeugte (Fehler-)Hinweis dann formatiert an den User ausgegeben werden. 
Hierzu wird aus Gründen der Testbarkeit an die Funktion \textit{printError} des Proxys \textit{Terminal} delegiert. \\
Diese Klasse befindet sich im Plugin-Layer (insb. im CLI-Plugin), da die konkrete Ausgabe an den User 
in Form von Text von dem CLI abhängt. Die high-level Fehlermeldungen die von den tieferen Schichten über 
Exceptions realisiert sind können von jedem Plugin anders verarbeitet und an den User weitergegeben werden. 
In der CLI funktioniert dies mit dem ErrorBuilder, während es mit einem denkbaren GUI-Plugin über bspw. 
ein Popup funktionieren könnte. Der ErrorBuilder ist lediglich für die Ausgabe an den Nutzer zuständig 
und enhält keine Fehler-Logik und gehört somit in die Plugin-Schicht. 

\begin{figure}[H]
	\centering
	\includegraphics[width=0.5\textwidth]{Bilder/ErrorBuilder_structure.pdf} 
	\caption{Beispielklasse der Plugin-Schicht: ErrorBuilder.}
	\label{fig:layer-ErrorBuilder}
\end{figure} 

\subsubsection{Schicht Domain:}

Der \textit{ResourceStash} ist fester Teil der Domain-Schicht und stellt hier ein Wrapper mit Methodennamen 
in Domänensprache für eine Resource-Deque dar. \autoref{fig:layer-ResourceStash} zeigt das zugehörte UML-Klassendiagramm.
Der ResourceStash beinhaltet und verwaltet alle Ressourcen, die der Spieler im Verlauf des Spiels ansammelt. 
Durch eine Katastrophe oder das Verlieren gegen ein Tier wird der ResourceStash zerstört (\textit{devastate})
und alle ungeschützten Ressourcen gelöscht. Das Erbauen eines \textit{Shack}s\footnote{nicht in der Abb.} 
erlaubt es die obersten $n \in [0;\inf)$ Ressourcen zu schützen (\textit{protectTopMostNResources}), 
sodass diese nach einem \textit{devastate} übrig bleiben. Zusätzlich können Ressourcen hinzugefügt,
konsumiert oder deren Vorhandensein überprüft werden. \textit{Camp} und \textit{Workbench} teilen 
sich eine Referenz auf denselben Stash. \\
Die Klasse ist als Teil des Kartenspiels (der Domäne) im Code natürlich in der Domain-Schicht angesiedelt und 
wird von anderen Domänenklassen genutzt. Es gibt keine Möglichkeit die Klasse in einer der anderen Schichten 
sinnvoll anzusiedeln.

\begin{figure}[H]
	\centering
	\includegraphics[width=1.05\textwidth]{Bilder/ResourceStash_structure.pdf} 
	\caption{Beispielklasse der Domain-Schicht: ResourceStash.}
	\label{fig:layer-ResourceStash}
\end{figure} 
\chapter{SOLID}

\section{Analyse Single-Responsibility-Principle (SRP)}

\subsubsection{Positiv-Beispiel:}

\begin{wrapfigure}{r}{0.40\textwidth}
	\centering
	\vspace{-30pt} % Manchmal möchte man den oberen Abstand selbst anpassen
	\includegraphics[width=0.30\textwidth]{Bilder/DiceParser_structure.pdf}
	\vspace{-10pt}
	% Das folgende ist ein Trick, um "Abbilgung x.y" in eine
	% eigene Zeile zu packen. Der Text zwischen [ und ] steht
	% im Abbildungsverzeichnis. Der Text darunter wird
	% tatsächlich angezeigt.
	\caption[UML-Diagramm von ase.plugin.cli.parsers.DiceParser.]{\unskip}
	UML-Diagramm von \textit{ase.plugin.cli.parsers.DiceParser}.
	\label{fig:srp-DiceParser}
\end{wrapfigure}

\autoref{fig:srp-DiceParser} zeigt das UML-Klassendiagramm des \textit{DiceParser}, welcher in der Plugin-Schicht
unter \texttt{plugin.cli.parsers} zu finden ist und das Positiv-Beispiel darstellt. 
Dieser implementiert das \textit{Parser}-Interface aus 
\texttt{plugin.cli}. Die einzige Aufgabe des DiceParsers ist es die Usereingabe zum Erstellen eines Würfelwurfs
aus Text zu parsen und das entsprechende Objekt zu kreieren. Hierzu wird der \textit{Regex}-Matcher, der 
bereits den Syntax überprüft hat an die \textit{parse}-Methode übergeben und daraus die Arugmente extrahiert, 
um den \textit{Roll} zu erstellen.

\subsubsection{Negativ-Beispiel:}

\autoref{fig:srp-RollHandler} zeig das UML-Klassendiagramm des \textit{RollHandler}, welcher in der Applikations-Schicht
die Würfelwürfe für die Rettungen (\textit{Endeavor}) \underline{und} Kämpfe (\textit{Encounter}) bearbeitet. 
Hierbei entscheidet die öffentliche \textit{handle}-Methode, ob es sich um einen Encounter oder ein Endeavor handelt
(dies ist allerdings eigentlich schon bei Aufruf dieser bekannt)
und führt die entsprechende private Methode aus. Die Klasse hat also zwei Aufgaben (Responsibilities). \\
Um dies zu lösen kann einfach die Klasse \textit{RollHandler} in zwei Klassen aufgeteilt werden - 
den \textit{EncounterHandler} und \textit{EndeavorHandler} - die nun jeweils nur noch genau eine Aufgabe haben 
und somit das SRP einhalten, wie \autoref{fig:srp-RollHandler-fixed} zeigt. 

\begin{figure}[H]
	\centering
	\includegraphics[width=1.\textwidth]{Bilder/RollHandler_structure.pdf} 
	\caption{UML-Diagramm von \textit{application.RollHandler}.}
	\label{fig:srp-RollHandler}
\end{figure} 

\begin{figure}[H]
	\centering
	\includegraphics[width=1.\textwidth]{Bilder/RollHandler_fixed_structure.pdf} 
	\caption{\autoref{fig:srp-RollHandler} aufgeteilt in \textit{EndeavorHandler} und \textit{EncounterHandler}.}
	\label{fig:srp-RollHandler-fixed}
\end{figure} 


\section{Analyse Open-Closed-Principle (OCP)}

\subsubsection{Positiv-Beispiel:}

Das Positiv-Beispiel zum OCP wird konstituiert durch das \textit{Command}-Interface aus \texttt{ase.plugin.cli} 
und dessen Implementationen (z.B. \textit{RollDxCommand}) aus \texttt{ase.plugin.cli.commands} wie gezeigt in \autoref{fig:ocp-rolldx}. 
Die Main-Klasse kennt nur das Command-Interface und ruft darauf die \textit{execute}-Methode auf, 
die dann in den unterschiedlichen Command-Implementationen verschiedene Wirkungen auf das übergebene 
\textit{Game} haben. Dies ist auch die Begründung, wieso das OCP hier efüllt wird: Um einen neuen Befehl 
zu implementieren muss lediglich eine weitere Klasse hinzugefügt werden, die das Command-Interface implementiert.
Es muss dazu keine der bestehenden Command-Klassen angepasst oder geändert werden. 

Das OCP ist hier sehr sinnvoll, da für neue Features sehr wahrscheinlich regelmäßig neue Commands hinzugefügt 
werden müssen und dies soll somit möglichst einfach umsetzbar sein und keinen bestehenden Code breaken. 
Zuvor wurden die unterschiedlichen Befehle über zahlreiche \texttt{switch}-Blöcke realisiert 
(siehe z.B. Commit 034a5c28), was Änderungen des Programmcodes an vielen Stellen nötig machte, 
um einen neuen Befehl hinzuzufügen. Außerdem wurden Runtime-Exceptions geworfen, wenn vergessen wurde 
den Code an einer Stelle anzupassen. Durch das neue System müssen \textit{keine} \underline{Änderungen} 
(geschlossen für Änderungen) an vielen Stellen mehr durchgeführt werden, 
sondern nur noch \underline{Additions} durchgeführt werden (offen für Erweiterung).  

\begin{figure}[H]
	\centering
	\includegraphics[width=0.7\textwidth]{Bilder/RollDxCommand_structure.pdf} 
	\caption{UML-Klassendiagramm vom \textit{RollDxCommand} und weiteren Commands, 
	wovon die meisten allerdings ausgelassen wurden der Übersichtlichkeit wegen.}
	\label{fig:ocp-rolldx}
\end{figure} 

\subsubsection{Negativ-Beispiel:}

\autoref{fig:ocp-CardInvalidator} zeigt das Negativ-Beispiel für das OCP. Die hier relevanten Klassen sind 
der \textit{CardInvalidator} und das \textit{Card}-Enum, welches eine \textit{CardCategory} und eine \textit{Resource} enthält. 
Der CardInvalidator zieht bisher eine Karte vom \textit{CardDeck} mit der \textit{draw}-Methode und invalidiert die Karte,
indem er deren Effekt einlöst. Dies wird mithilfe eines \texttt{switch}-Statements über die CardCategory der Karte entschieden.
Wenn nun also eine neue Karte mit einem neuen Karteneffekt hinzugefügt werden soll, muss also die \textit{draw}-Methode 
\underline{verändert} werden. Das OCP ist also nicht gewahrt. \\
\autoref{fig:ocp-CardInvalidator-fixed} zeigt eine mögliche Lösung, um das OCP einzuhalten: 
Das Card-Enum wurde mit einem Card-Interface mit der \textit{invalidate}-Methode ersetzt 
und die Karten (Metal, Wood, Tiger, etc.) sind nun Implementationen des Card-Interface und \textit{CardCategory} wurde entfernt.
Somit kann der CardInvalidator in 
\textit{draw} einfach die invalidate-Methode der gezogenen Karte aufrufen und die Karte führt selbst ihren Effekt aus. 
Somit können in Zukunft beliebig neue Karten eingeführt werden, für die einfach nur eine Implementation des Card-Interface 
geschrieben werden muss. Es ist keine Veränderung bestehenden Codes mehr nötig und somit wird das OCP mit dieser Lösung eingehalten.


\begin{figure}[H]
	\centering
	\includegraphics[width=0.5\textwidth]{Bilder/CardInvalidator_structure.pdf} 
	\caption{UML-Klassendiagramm von \textit{CardInvalidator} und \textit{Card}.}
	\label{fig:ocp-CardInvalidator}
\end{figure} 

\begin{figure}[H]
	\centering
	\includegraphics[width=0.55\textwidth]{Bilder/CardInvalidator_fixed_structure.pdf} 
	\caption{\autoref{fig:ocp-CardInvalidator} mit Card-Interface und Implementationen statt Card-Enum und CardCategory. 
	Aus Übersichtlichkeit wurden nicht alle Implementationen eingezeichnet.}
	\label{fig:ocp-CardInvalidator-fixed}
\end{figure} 


\section{Analyse Interface-Segreggation-Principle (ISP)}

\subsubsection{Positiv-Beispiel:} 

\autoref{fig:isp-Buildable} zeigt wie die Interfaces \textit{Buildable}, \textit{Tool}, \textit{Building} und \textit{Rescue} 
das ISP einhalten. Die Abhängigkeiten von anderen Klassen auf Tool, Building und Rescue sind aus Übersichtlichkeitsgründen ausgelassen.
Alle Interfaces enthalten nur eine Methode und alle Tool-, Building- und Rescue-Implementationen sind auch Buildables 
aber nicht anders herum. So können die Interfaces sehr flexibel angewandt werden und das ISP ist gewahrt. 

\begin{figure}[H]
	\centering
	\includegraphics[width=0.9\textwidth]{Bilder/Buildable_structure.pdf} 
	\caption{UML-Diagramm von Buildable, Endeavor, Tool und Building und deren Implementationen.}
	\label{fig:isp-Buildable}
\end{figure} 

\subsubsection{Negativ-Beispiel:}

\begin{figure}[H]
	\centering
	\includegraphics[width=0.5\textwidth]{Bilder/Deck_structure.pdf} 
	\caption{UML-Diagramm von \textit{Deck} und dessen Implementationen.}
	\label{fig:isp-Deck}
\end{figure} 


\autoref{fig:isp-Deck} zeigt das \textit{Deck}-Interface - das einzige Interface in der Applikation mit mehr als einer Methode.
Zunächst ist es allerdings diskutabel, ob es sich hierbei wirklich um ein Negativ-Beispiel handelt, da es sich bei dem Deck-Interface 
um eine Datenstruktur handelt, die besondere Eigenschaften aufweisen soll, wie z.B. dass der ursprüngliche Zustand der Datenstruktur 
bei Aufruf der \textit{reset}-Methode wieder hergestellt werden soll. Alle Methoden des Deck-Interfaces werden also 
gesammelt und gemeinsam benötigt, damit eine Klasse von der Abstraktion des Deck-Interfaces abhängen kann und den nötigen 
Funktionsumfang verwenden kann. Hier kann also erstmal das Deck-Interface nicht aufgeteilt werden. 
Außerdem hängt das Deck-Interface von dem \textit{Iterator}-Interface ab, sodass alle Implementationen von Deck auch 
automatisch Iterator implementieren müssen. Diese Funktion ist aber ebenfalls für ein Deck gewünscht - es soll möglich sein, 
mit einer Schleife über die Elemente des Deck zu iterieren. Damit ist also auch nicht das Iterator-Interface abspaltbar. \\
Allerdings wird es unmittelbar klar, dass das ISP hier eindeutig und ohne Diskussion gebrochen wird, wenn betrachtet wird, 
dass Deck ebenfalls das \textit{Serializable}-Interface beinhaltet, wodurch automatisch alle Implementationen das 
Serializable-Interface implementieren, was natürlich absolut \textit{keinen} Sinn ergibt. Es ist sogar gefährlich: 
Beim Implementieren des Deck-Interface wird dem Programmierer nicht klar, dass er die Implementation implizit und ohne Kenntnis als 
Serializable markiert, was ein großes Problem darstellen kann. \\ 
\autoref{fig:isp-Deck-fixed} zeigt nun, dass dieses Problem einfach behoben werden kann, indem die Implementationen explizit und direkt
Serializable implementieren und so bewusst die jeweilige Implementation als Serializable markieren. Damit ist das ISP wieder gewahrt,
da das Deck so weit wie sinnvoll möglich (Erläuterung s.o.) verkleinert wurde und die Implementationen nun von mehreren kleineren Interfaces abhängen,
anstelle von einem größeren abhängen. 

\begin{figure}[H]
	\centering
	\includegraphics[width=0.5\textwidth]{Bilder/Deck_fixed_structure.pdf} 
	\caption{\autoref{fig:isp-Deck} mit abgespaltetem \textit{Serializable}-Interface.}
	\label{fig:isp-Deck-fixed}
\end{figure} 


\chapter{Weitere Prinzipien} 

\section{Analyse GRASP: Geringe Kopplung}

\subsubsection{Positiv-Beispiel:}

\autoref{fig:coupling-PersistenceWriter} zeigt das Positiv-Beispiel zur geringen Kopplung. \\
Das Interface \textit{ase.application.PersistenceWriter} entkoppelt \textit{ase.application.Game} von 
\textit{ase.plugin.localpersistence.SerializationFilePersistor}. In Game soll es die Möglichkeit geben den aktuellen 
Spielstand zu speichern, damit dieser in Zukunft wieder geladen werden kann und das Spiel fortgesetzt werden kann. 
SerializationFilePersistor bietet diese Möglichkeit, in dem es den aktuellen Spielstand als Java-Objekt serialisiert 
und als Datei im lokalen Dateisystem speichert. Es darf aber auf keinen Fall eine direkte Abhängigkeit von 
Game zu SerializationFilePersistor bestehen, da dies die Dependency Rule brechen würde und außerdem zu einer starken 
Kopplung führen würde. Mit der Abstraktion durch das Interface ist es möglich in Zukunft andere Speichermethoden einzuführen,
ohne dass sich etwas für Game ändert, wodurch sich die positiven Effekte der geringen Kopplung entfalten. 

\begin{figure}[H]
	\centering
	\includegraphics[width=0.5\textwidth]{Bilder/PersistenceWriter_structure.pdf} 
	\caption{UML-Diagramm von \textit{ase.application.PersistenceWriter}.}
	\label{fig:coupling-PersistenceWriter}
\end{figure} 

\subsubsection{Negativ-Beispiel:}

\autoref{fig:coupling-Workbench} zeigt das Negativ-Beispiel, da eine sehr starke Kopplung zwischen \textit{Workbench} 
und \textit{ResourceStash} als direkte Abhängigkeit vorliegt\footnote{Das gleiche Problem besteht auch 
zwischen \textit{Camp} und \textit{Workbench} aber hier wird nur \underline{ein} Negativ-Beispiel gewünscht und behoben.}. \\
Aufgabe der Workbench ist es zu überprüfen, welche \textit{CraftingPlan}s mit den vorhandenen Ressourcen im 
ResourceStash herstellbar sind und Gegenstände mit der \textit{build}-Methode herzustellen, indem dafür die 
nötigen Ressourcen vom ResourceStash verwendet werden. Aufgabe des ResourceStash ist es in erster Linie eine Datenstruktur
zur Lagerung von Ressourcen bereitzustellen. Die Workbench nutzt dabei nur die Methoden \textit{consumeResources} 
und \textit{hasResources} des ResourceStash. \\
Wie \autoref{fig:coupling-Workbench-fixed} zeigt, kann die hohe Kopplung einfach aufgelöst werden, 
indem zwischen Workbench und ResourceStash ein Interface eingezogen wird, welches die Methoden bereitstellt, 
die Workbench zum verichten seiner Aufgaben benötigt. Dadurch könnte in Zukunft ResourceStash mit einer anderen 
Datenstruktur ausgetauscht werden, ohne dass die Workbench davon etwas mitbekommen würde, aufgrund der nun geringen Kopplung.

\begin{figure}[H]
	\centering
	\includegraphics[width=0.5\textwidth]{Bilder/Workbench_structure.pdf} 
	\caption{UML-Diagramm von \textit{ase.domain.crafting.Workbench}.}
	\label{fig:coupling-Workbench}
\end{figure} 

\begin{figure}[H]
	\centering
	\includegraphics[width=0.5\textwidth]{Bilder/Workbench_fixed_structure.pdf} 
	\caption{\autoref{fig:coupling-Workbench} mit eingezogenem \textit{ResourceConsumer}-Interface 
	zur Auflösung der Kopplung.}
	\label{fig:coupling-Workbench-fixed}
\end{figure} 

\section{Analyse GRASP: Hohe Kohäsion}

\autoref{fig:cohesion-ResourceRequirement} zeigt das UML-Diagramm der Klasse \textit{ResourceRequirement}, 
welches eine sehr hohe Kohäsion hat. Es handelt sich um ein Datentupel aus (Ressource, Menge), 
welches im \textit{CraftingPlan}-Enum verwendet wird, um die benötigten Ressourcen für das jeweilige 
Buildable anzugeben. Die Kohäsion ist sehr hoch, da zu einer Bedarfsangabe immer gehört, welcher Gegenstand 
(\textit{resource}) benötigt wird und wie viel davon (\textit{amount}). Die beiden Angaben sind im Rahmen der 
Bedarfsangabe semantisch maximal zusammenhängend und können inhaltlich nicht voneinander getrennt werden.   

\begin{figure}[H]
	\centering
	\includegraphics[width=0.55\textwidth]{Bilder/ResourceRequirement_structure.pdf} 
	\caption{UML-Diagramm von \textit{ase.domain.crafting.ResourceRequirement}.}
	\label{fig:cohesion-ResourceRequirement}
\end{figure} 


\section{Don't Repeat Yourself (DRY)}

In Commit \texttt{e246d77b} wurde der folgende duplizierte Code aus der \textit{Game}-Klasse gebündelt: \\
\texttt{state.setPhase(GamePhase.END);} \\
\texttt{state.setStatus(GameStatus.ENDED);} \\
\texttt{state.setResult(X);} \\
\autoref{code:dry-before} zeigt den Code ein Commit vorher (ID: \texttt{b4399813}).
\autoref{code:dry-after} zeig den Code mit der neuen Methode \textit{endGameWith}, die die das Code-Duplikat 
an allen drei Stellen entfernt. \\
Es ist wichtig den Code an der Stelle anzupassen, um bei Änderungen der Game-Schlusslogik, keine 
Stelle ausversehen zu vergessen, was zu Bugs führen könnte. Ein Beispiel wäre, dass ein zusätzlicher Zustand 
gesetzt werden muss, Observern ein Statusupdate mitgeteilt werden soll, oder ein anderer Zustand gesetzt 
werden soll, weil sich das Modell ändert. \\
Durch das Entfernen der Duplikate ist der Code kürzer, einfacher, besser lesbar (sprechender Name der Methode anstatt 
willkürlich erscheinende Statusänderungen) und besser wartbar aus den zuvor genannten Gründen.   


\lstinputlisting[
	label=code:dry-before,    % Label; genutzt für Referenzen auf dieses Code-Beispiel
	caption=DRY-Code der \textit{Game}-Klasse zum Commit \texttt{b4399813}.,
	captionpos=b,               % Position, an der die Caption angezeigt wird t(op) oder b(ottom)
	style=EigenerJavaStyle,   % Eigener Style der vor dem Dokument festgelegt wurde
	firstline=1,                % Zeilennummer im Dokument welche als erste angezeigt wird
	lastline=100                 % Letzte Zeile welche ins LaTeX Dokument übernommen wird
]{Quellcode/dry-before.java}

\lstinputlisting[
	label=code:dry-after,    % Label; genutzt für Referenzen auf dieses Code-Beispiel
	caption=DRY-Code der \textit{Game}-Klasse zum Commit \texttt{e246d77b}.,
	captionpos=b,               % Position, an der die Caption angezeigt wird t(op) oder b(ottom)
	style=EigenerJavaStyle,   % Eigener Style der vor dem Dokument festgelegt wurde
	firstline=1,                % Zeilennummer im Dokument welche als erste angezeigt wird
	lastline=100                 % Letzte Zeile welche ins LaTeX Dokument übernommen wird
]{Quellcode/dry-after.java}



\chapter{Unit Tests} \label{sec:results}


\section{10 Unit Tests}

\autoref{tab:tests} zeigt die geforderten Tests. 

\newcounter{rowcounter}
\newcommand\rownumber{\stepcounter{rowcounter}\arabic{rowcounter}}

\newcolumntype{b}{X}
\newcolumntype{s}{>{\hsize=.5\hsize}X}

\begin{table}[H]
	\centering
	\begin{tabularx}{\textwidth}{rs|b}
		& \textbf{Unit Test} & \textbf{Beschreibung} \\
		\midrule
		\rownumber & RollIntegerTest \#fromNumberCappedTest & 
		Testet statische \textit{fromNumberCapped}-Funktion der \textit{RollInteger}-Klasse, ob das richtige, ggf. beschränkte, Ergebnis 
		erzeugt wird, bzw. illegale Eingaben abgewiesen werden. \\ 
		\rownumber & Roll\#raiseRollByTest &
		Testet \textit{raiseRollBy}-Methode der \textit{Roll}-Klasse, ob die Erhöhung den Domänenregeln entsprechend durchgeführt wird. \\
		\rownumber & RollHandlerTest \#encounter &
		Testet encounter-Methode des RollHandler auf verschiedenen legalen Input von gemocktem Bonusschaden, Tieren und Würfelwürfen. \\
		\rownumber & RollHandlerTest \#endeavor & Testet endeavor-Methode des RollHandler mit RescueMock mit verschiedenen Ergebnissen. \\
		\rownumber & CollectionStringerTest \#collectionToStringTest &
		Testet \textit{collectionToString}-Methode der \textit{CollectionStringer}-Klasse,
		ob die String-Konvertierung insb. mit den Zeilenumbrüchen ordentlich funktioniert. \\
		\rownumber & DiceParserTest \#parseTest &
		Testet \textit{parse}-Methode der \textit{DiceParser}-Klasse,
		ob die verschiedenen Würfeltypen und Augenzahlen, sowie die Random-Roll-Funktion für die jeweiligen Würfel korrekt parsen. \\
		\rownumber & NonNegativeIntegerTest \#addTest & 
		Testet add-Methode von NonNegativeInteger gründlich, ob richtiges Ergebnis erzeugt wird bzw. illegale Eingaben abgewiesen werden \\ 
		\rownumber & NonNegativeIntegerTest \#newNonNegativeIntegerTest & 
		Testet Konstruktor von NonNegativeInteger ungründlich, ob Eingabe zugelassen wird. \\ 
		\rownumber & ResourceStashTest \#devastateTest &
		Testet \textit{devastate}-Methode der \textit{ResourceStash}-Klasse,
		ob die Ressourcen bis auf genau die obersten $n$ gelöscht werden. \\
		\rownumber & ResourceStashTest \#hasResourcesTest &
		Testet \textit{hasResources}-Methode der \textit{ResourceStash}-Klasse,
		ob verschiedene \textit{ResourceRequirements} immer korrekt behandelt werden. \\

	\end{tabularx}
	\caption{Zehn Unit Tests mit Namen (ausgehend von \textit{de.dhbw.karlsruhe.ase}) und Beschreibung. (Packages ausgelassen aus Platzgründen).}
	\label{tab:tests}
\end{table}


\section{ATRIP: Automatic}

\textit{Automatic} verlangt, dass mit minimalem Aufwand die Ausführung der Unit Tests angestoßen werden kann. Hierfür 
wird das Unit-Testing-Framework \textit{JUnit} verwendet, wonach Methoden erstellt werden können, die mit \textit{@Test} annotiert sind, 
wodurch sie durch JUnit ausgeführt werden, wenn über die IDE oder Maven ein entsprechender Befehl erfolgt. Zur Unterstützung 
der automatisierten Ausführung sind die Unit Tests so geschrieben, dass sie \textit{Repeatable} und \textit{Independent} sind, 
damit es keine Probleme bei der (wiederholten) automatischen Ausführung der Tests in beliebiger Reihenfolge gibt. Außerdem haben die 
Tests keine externen Abhängigkeiten - auch nicht zu diversen Dateien mit Testinhalten oder Ähnliches. Nach Durchführen der Tests 
gibt JUnit eine Übersicht über die erfolgreichen und ggf. fehlgeschlagenen Tests. Es sind keine Nutzereingaben notwendig. 
Es muss nichts weiter konfiguriert 
werden, um die Tests durchzuführen. Der Konsolenbefehl \texttt{mvn test} genügt direkt nach dem Klonen des Repositorys. 

\section{ATRIP: Thorough}

\subsubsection{Positiv-Beispiel}

\autoref{code:thorough-positive} zeigt das Positiv-Beispiel zu \textit{thorough} Test-Code. Hier wird 
die \textit{add}-Methode der \textit{NonNegativeInteger}-Klasse getestet, die auf einen bestehenden 
NonNegativeInteger einen Java-Integer-Primitive addiert, der sowohl negativ als auch positiv sein kann (solange 
die Summe/Differenz nicht-negativ ist) und das Ergebnis als neuen NonNegativeInteger zurückgibt. Ist das 
Resultat negativ, soll eine \textit{IllegalArgumentException} geworfen werden. \\
Der Test ist thorough, weil durch den \textit{ParameterizedTest} und die dazugehörige \textit{MethodSource} \textit{add}
Stichproben getestet werden, die die Verteilung der möglichen Eingaben \textit{gründlich} abdecken sollten, 
da positive Standardfälle, Edge-Cases und pathologische Eingaben getestet werden.    

\lstinputlisting[
	label=code:thorough-positive,    % Label; genutzt für Referenzen auf dieses Code-Beispiel
	caption=\underline{\textit{Thorough}} Test-Code zur \textit{add}-Methode der \textit{NonNegativeInteger}-Klasse.,
	captionpos=b,               % Position, an der die Caption angezeigt wird t(op) oder b(ottom)
	style=EigenerJavaStyle,   % Eigener Style der vor dem Dokument festgelegt wurde
	firstline=13,                % Zeilennummer im Dokument welche als erste angezeigt wird
	lastline=43                 % Letzte Zeile welche ins LaTeX Dokument übernommen wird
]{Quellcode/thorough.java}

\subsubsection{Negativ-Beispiel}

\autoref{code:thorough-negative} zeigt das Negativ-Beispiel zu \textit{thorough} Test-Code. Hier wird 
der Konstruktor der \textit{NonNegativeInteger}-Klasse getestet, der über eine \textit{IllegalArgumentException} 
sicherstellen muss, dass der übergebene Java-Integer-Primitive nicht negativ, also $\geq 0$ ist. \\
Der Test ist \textit{nicht} thorough, weil durch den Test nur ein einziger positiver Fall ($5 \geq 0$) getestet 
wird, keine Edge-Cases (Eingabe = \textit{Integer.MAX\_INT}, Eingabe = $0$) oder pathologischen Eingaben 
(z.B. Eingabe = $-3$) getestet werden. Die Hauptfunktionalität - also das Ablehnen von negativen Eingaben zum 
Einhalten des Vertrags - wird so nicht ein mal getestet und damit eigentlich nur triviales Konstruktorverhalten.

\lstinputlisting[
	label=code:thorough-negative,    % Label; genutzt für Referenzen auf dieses Code-Beispiel
	caption=\underline{Nicht} \textit{thorough} Test-Code zum Konstruktor der \textit{NonNegativeInteger}-Klasse.,
	captionpos=b,               % Position, an der die Caption angezeigt wird t(op) oder b(ottom)
	style=EigenerJavaStyle,   % Eigener Style der vor dem Dokument festgelegt wurde
	firstline=45,                % Zeilennummer im Dokument welche als erste angezeigt wird
	lastline=50                 % Letzte Zeile welche ins LaTeX Dokument übernommen wird
]{Quellcode/thorough.java}

\section{ATRIP: Professional}

\subsubsection{Positiv-Beispiel}

\autoref{code:pro-positive} zeigt das Positiv-Beispiel zu \textit{professional} Test-Code. Hier wird die 
\textit{raiseRollBy}-Methode der \textit{Roll}-Klasse getestet, die einen bestehenden \textit{Roll} 
um einen \textit{NonNegativeInteger} erhöhen soll, aber nur bis zu dem maximalen legalen Wert des bestehenden Rolls, 
der abhängig von dessen \textit{DiceType} ist, und das Resultat als neuen Roll zurückgeben soll. \\
Der Test ist professionell da keinerlei Code-Duplication vorliegt und das SRP eingehalten ist, 
da der Test als \textit{ParameterizedTest} all die Test-Cases, die durch die \textit{MethodSource} gegeben sind, ausführt.
Durch den ParameterizedTest ist auch leicht zu erkennen, welche Test-Cases erfolgreich waren und welche nicht. 
Die Variablen- und Methodennamen sind leicht verständlich und der Code ist gut lesbar. 
Es bestehen auch keine sonstigen Code-Smells. Zudem testet der Test eine relevante und nicht-triviale Methode, 
die wichtig für das Einhalten der Domänenregeln ist und deswegen korrekt sein muss, ansonsten würden Bugs 
verteilt in der Codebase auftreten.  

\lstinputlisting[
	label=code:pro-positive,    % Label; genutzt für Referenzen auf dieses Code-Beispiel
	caption=\underline{Professioneller} Test-Code zur \textit{raiseRollBy}-Methode der \textit{Roll}-Klasse.,
	captionpos=b,               % Position, an der die Caption angezeigt wird t(op) oder b(ottom)
	style=EigenerJavaStyle,   % Eigener Style der vor dem Dokument festgelegt wurde
	firstline=13,                % Zeilennummer im Dokument welche als erste angezeigt wird
	lastline=61                 % Letzte Zeile welche ins LaTeX Dokument übernommen wird
]{Quellcode/professional.java}

\subsubsection{Negativ-Beispiel}

\autoref{code:pro-negative} zeigt das Negativ-Beispiel zu \textit{professional} Test-Code. Hier wird die 
statische \textit{fromNumberCapped}-Funktion der \textit{RollInteger}-Klasse getestet, die aus einem Java-Integer-Primitive und 
einem \textit{DiceType} einen neuen RollInteger erstellen soll, dessen Wert dem Primitive entspricht, aber maximal 
dem Maximalwert des übergebenen DiceType. Zahlen $< 1$ sollen durch eine \textit{IllegalArgumentException} abgefangen werden. \\
Der Test ist unprofessionell aus folgenden Gründen:
\begin{enumerate}
	\item Es besteht extrem viel Code-Duplication, da der Test wohl durch Copy-Paste erzeugt wurde. 
	Dies sollte durch eine Method-Extraction behoben werden. 
	\item Alle Test-Cases stehen in der einen Test-Methode ($\rightarrow$ SRP verletzt), wodurch es schwieriger wird bei 
	teilweise fehlschlagenden Assertions zu erkennen, wo der Fehler liegt. Die Test-Cases sollten in eigene Tests ausgelagert werden, 
	sodass das SRP gewahrt wird. Zusammen mit dem vorherigen Punkt sollte also ein \textit{ParameterizedTest} mit \textit{MethodSource}
	für die Test-Cases erstellt werden.
	\item Es werden die temporären Variablen \textit{inputInteger} und \textit{inputDiceType} nicht konsequent angewandt, 
	sondern trotzdem noch magische Konstanten (s. z.B. Zeile 15, wo eine $2$ steht, statt \textit{inputInteger}). 
	\item Bei den Tests, bei denen der \textit{inputInteger} das Maximum des Würfeltyps überschreitet (z.B. Block Zeile 17 ff.), 
	sollte keine magische Konstante (im Beispiel $6$ für \textit{DiceType.SIX\_SIDED}) als erwarteter Wert verwendet werden,
	sondern \textit{inputDiceType.integerRepresentation}, welche das Maximum darstellt.
	\item In Zeile 35 steht ein Multiline-Kommentar mitten im Code, um eine magische Konstante zu erklären. Um diesen Code-Smell aufzuheben,
	sollte die Konstante \textit{Integer.MIN\_VALUE} eingefügt werden.
\end{enumerate}

\lstinputlisting[
	label=code:pro-negative,    % Label; genutzt für Referenzen auf dieses Code-Beispiel
	caption=\underline{\textbf{Un}professioneller} Test-Code zur statischen \textit{fromNumberCapped}-Funktion der \textit{RollInteger}-Klasse.,
	captionpos=b,               % Position, an der die Caption angezeigt wird t(op) oder b(ottom)
	style=EigenerJavaStyle,   % Eigener Style der vor dem Dokument festgelegt wurde
	firstline=8,                % Zeilennummer im Dokument welche als erste angezeigt wird
	lastline=46                 % Letzte Zeile welche ins LaTeX Dokument übernommen wird
]{Quellcode/unprofessional.java}

\section{Code Coverage}

\begin{table}[H]
	\centering
	\begin{tabular}{l|l|l}
	Line & Method & Class \\
	\midrule
	91\% (611/671) & 95\% (237/249) & 100\% (67/67) 
	\end{tabular}
	\caption{Code-Coverage der Tests relativ und absolut nach Statement-, Method- und Class-Coverage.}
	\label{tab:coverage}
\end{table}

\autoref{tab:coverage} zeigt die Code-Coverage aller Tests im Projekt nach Statement-, Method- und Class-Coverage sowohl als relativer (prozentualer) Wert, 
als auch als absoluter Wert. Der Wert von 91\% Statement-Coverage wird als gut für das konkrete Projekt angesehen. Insbesondere, wenn beachtet wird, 
welche Zeilen nicht abgedeckt sind: zumeist das Werfen von Runtime-Exceptions oder das Behandeln von Exceptions im Plugin-Code in Form von 
Fehlerausgaben an den Nutzer. Es gibt zwar kein allgemein anerkannten Wert für eine \enquote{gute} Code-Coverage, aber für ein brettspielartiges 
Kartenspiel ist dies ein ausreichender Wert, da keine größeren Bugs vorhanden sein sollten und das Spiel im Großen und Ganzen gut funktionieren sollte,
abgesehen von möglicherweise ein paar wenigen kleineren Bugs. Insbesondere die Blackbox-Integration-Tests sichern einen normalen Spielverlauf zu. 
Da es sich außerdem um kein Online-Spiel handelt und es keine Leaderboards gibt, muss das Spiel auch nicht zu sehr auf diverse kleinere Exploits 
oder Glitches getestet werden. Für andere, sicherheitskritische Software wäre eine Code-Coverage von 91\% sicher unzureichend, aber für dieses 
Anwendung ist es ausreichend.       


\section{Fakes und Mocks}

\subsubsection{CampMock}

\autoref{fig:CampMock} zeigt das UML-Diagramm eines Mocks zum Testen der \textit{encounter}-Methode des \textit{RollHandler}. 
Der \textit{CampMock} ist realisiert als Subklasse der zu mockenden Oberklasse \textit{Camp}, wobei CampMock die relevanten 
zu mockenden Methoden überschreibt. Konkret ist das hier lediglich die \textit{getBonusDamage}-Methode, die im Mock das private
Attribut \textit{bonusDamage} zurückgibt, welches durch den Konstruktor gesetzt wird. \\ 
Grund für den Mock ist, dass in der encounter-Methode auf die getBonusDamage-Methode von Camp zurückgegriffen wird, um zu bestimmen,
ob ein Spieler gegen ein Tier gewinnt oder nicht. Das Setup, um ein Camp zu erzeugen, wobei die getBonusDamage-Methode tatsächlich 
einen gewünschten Wert zurückgibt wäre aber ohne Mock unverhältnismäßig schwierig, da zunächst genügend Ressourcen in das Camp 
eingefüllt werden müssten, dass es reicht ein \textit{Tool} zu bauen, wovon der bonusDamage genommen wird. Außerdem könnten dann 
nur bereits über Tools implementierte Werte für getBonusDamage getestet werden und nicht beliebige. Daher wird hier ein Mock benötigt.     

\begin{figure}[H]
	\centering
	\includegraphics[width=1\textwidth]{Bilder/CampMock_structure.pdf} 
	\caption{UML-Diagramm des \textit{CampMock}, der zur Testung der \textit{encounter}-Methode des \textit{RollHandler} benötigt wird.} 
	\label{fig:CampMock}
\end{figure} 

\subsubsection{RescueMock}

\autoref{fig:RescueMock} zeigt das UML-Diagramm eines Mocks zum Testen der \textit{endeavor}-Methode des \textit{RollHandler}. 
Der \textit{RescueMock} ist realisiert als Implementation des zu mockenden Interfaces \textit{Rescue}, wobei RescueMock die endeavor-Methode 
so überschreibt, dass ein Boolean-Wert zurückgegeben wird, der zuvor über den Konstruktor von RescueMock injiziert wurde. \\ 
Grund für den Mock ist, dass in der endeavor-Methode auf die getCurrentEndeavor-Methode von Camp zurückgegriffen wird, um auf dem 
zurückgegebenen Interface die endeavor-Methode aufzurufen, um zu überprüfen, ob der Spieler von der Insel entfliehen kann. 
Hierfür wird also in Camp der RescueMock injiziert, sodass getCurrentEndeavor den Mock zurückgibt. 
Das Setup, um ein Camp zu erzeugen, welches eine Rescue hält, die einen gewünschten Wert zurückgibt wäre aber ohne Mock unverhältnismäßig schwierig,
da zunächst genügend Ressourcen in das Camp eingefüllt werden müssten, dass es reicht eine \textit{Rescue} zu bauen, 
deren endeavor-Methode den gewünschten Wert zurück gibt. Außerdem könnten dann 
nur bereits über Rescues implementierte Verhalten für Rescue\#endeavor getestet werden und nicht beliebige. Zusätzlich würde 
der Test dann auch die Rescue\#endeavor-Methode der konkreten Rescues testen und nicht nur isoliert die RollHandler\#endeavor-Methode.
Daher wird hier ein Mock benötigt.     

\begin{figure}[H]
	\centering
	\includegraphics[width=1\textwidth]{Bilder/RescueMock_structure.pdf} 
	\caption{UML-Diagramm des \textit{RescueMock}, der zur Testung der \textit{endeavor}-Methode des \textit{RollHandler} benötigt wird.}
	\label{fig:RescueMock}
\end{figure} 
\chapter{Domain Driven Design} \label{sec:ddd}
\chapter{Refactoring}

\section{Code-Smells} \label{sec:smells}

\subsubsection{Long-Class} 

\autoref{code:long-class-before} und \autoref{code:long-class-after} zeigen den Code eines Long-Class-Code-Smells 
vor (Commit \texttt{174283ab}) und nach (Commit \texttt{b0ce98af}) der Behebung mittels Aufteilen der Klasse 
in drei neue Klassen, die aber jeweils ein gemeinsames Interface (\textit{Parser}) implementieren. \\
Vorher gab es eine \textit{ArgumentParser}-Klasse, die als Methoden-Sammlung für das Parsen von 
\textit{CraftingPlan}s (\textit{parseConstructible}), \textit{Dice} (\textit{parseDice}) 
und \textit{Card}s (\textit{parseCards}) zuständig war. Hierdurch war die Klasse nicht SRP- oder OCP-konform 
und trug den Long-Class-Code-Smell. \\
Dies wurde in \autoref{code:long-class-after} gelöst, 
indem die ArgumentParser-Klasse aufgeteilt wurde in \textit{CardParser}, \textit{CraftingPlanParser} und \textit{DiceParser},
die jeweils das \textit{Parser}-Interface implementieren und somit nur die \textit{parse}-Methode haben. 
(Außerdem wurden die \textit{switch}-Statements gleich mit aufgelöst, aber dieser Code-Smell wird hier nicht betrachtet).
Dadurch wahrt die neue Lösung das SRP und OCP und es ist keine long-class mehr. 

\lstinputlisting[
	label=code:long-class-before,    % Label; genutzt für Referenzen auf dieses Code-Beispiel
	caption=Long-Class-Code-Smell der \textit{ArgumentParser}-Klasse zum Commit \texttt{174283ab}.,
	captionpos=b,               % Position, an der die Caption angezeigt wird t(op) oder b(ottom)
	style=EigenerJavaStyle,   % Eigener Style der vor dem Dokument festgelegt wurde
	firstline=1,                % Zeilennummer im Dokument welche als erste angezeigt wird
	lastline=100                 % Letzte Zeile welche ins LaTeX Dokument übernommen wird
]{Quellcode/long-class-before.java}


\lstinputlisting[
	label=code:long-class-after,    % Label; genutzt für Referenzen auf dieses Code-Beispiel
	caption=Long-Class-Code-Smell behoben durch Aufteilung der Klassen zum Commit \texttt{b0ce98af}.,
	captionpos=b,               % Position, an der die Caption angezeigt wird t(op) oder b(ottom)
	style=EigenerJavaStyle,   % Eigener Style der vor dem Dokument festgelegt wurde
	firstline=1,                % Zeilennummer im Dokument welche als erste angezeigt wird
	lastline=100                 % Letzte Zeile welche ins LaTeX Dokument übernommen wird
]{Quellcode/long-class-after.java}

\subsubsection{Switch-Statement} 

\autoref{code:switch-before} und \autoref{code:switch-after} zeigen den Code eines Switch-Statement-Code-Smells 
vor (Commit \texttt{dd2a39d2}) und nach (Commit \texttt{a0e99692}) der Behebung mittels Einführung einer 
Map, die das Zählen dynamisch übernimmt. \\ 
Vorher gab es nur die \textit{CardDeck}-Klasse, die mit der \textit{isValid}-Methode überprüft, 
ob die Karten-Konfiguration (also die Anzahl der verschiedenen Karten im Deck) zulässig ist. 
Das Zählen der Karten funktionierte über ein Switch-Statement, welches einen explizit für diesen 
Kartentyp vorgesehenen Zähler inkrementierte, wenn die Karte bei Iteration über die Sammlung von eben diesem Typ war.
Der Vergleich auf Validität erfolgte über eine Kaskade von verundeten Gleichheitstests mit den erwünschten Werten. 
Diese Methode war furchtbar unerweiterbar. \\
Die Lösung in \autoref{code:switch-after} funktioniert so: Es wurde das neue VO \textit{CardDeckConfiguration} 
eingeführt, welches eine immutable Map von Card -> Integer mappt und ermöglicht zu überprüfen, ob zwei 
CardDeckConfigurations wertetechnisch gleich sind (\textit{record}'s implizites \textit{equals}). Zur 
besseren Vergleichbarkeit hat CardDeckConfiguration die \textit{withoutZeroOccurrences}-Methode, um Karten, 
die null Mal vorkommen, zu entfernen. CardDeck hat nun das Attribut \textit{config}, welches die erwünschte 
CardDeckConfiguration beinhaltet. Der Switch in \textit{isValid} wurde nun ersetzt, indem durch einen 
Loop über das CardDeck die Anzahl an Karten eines Typs dynamisch über eine Map-Struktur gezählt werden (Kartentyp 
ist der Schlüssel) und dann schließlich die Ist- und Soll-Konfiguration über ein \textit{equals}-Aufruf verglichen 
werden. Hierdurch wurde der Switch-Code-Smell entfernt und CardDeck ist nun OCP-konform, da nichts an der Klasse 
(also der frühere Switch und die Zähler pro Kartentyp und deren Vergleich) geändert werden muss, wenn ein neuer 
Kartentyp hinzugefügt wird.   

\lstinputlisting[
	label=code:switch-before,    % Label; genutzt für Referenzen auf dieses Code-Beispiel
	caption=Switch-Statement-Code-Smell der \textit{CardDeck}-Klasse zum Commit \texttt{dd2a39d2}.,
	captionpos=b,               % Position, an der die Caption angezeigt wird t(op) oder b(ottom)
	style=EigenerJavaStyle,   % Eigener Style der vor dem Dokument festgelegt wurde
	firstline=1,                % Zeilennummer im Dokument welche als erste angezeigt wird
	lastline=100                 % Letzte Zeile welche ins LaTeX Dokument übernommen wird
]{Quellcode/switch-before.java}


\lstinputlisting[
	label=code:switch-after,    % Label; genutzt für Referenzen auf dieses Code-Beispiel
	caption=Switch-Statement-Code-Smell behoben durch neue \textit{DeckConfiguration}-Klasse und Map zum Zeitpunkt des Commit \texttt{a0e99692}.,
	captionpos=b,               % Position, an der die Caption angezeigt wird t(op) oder b(ottom)
	style=EigenerJavaStyle,   % Eigener Style der vor dem Dokument festgelegt wurde
	firstline=1,                % Zeilennummer im Dokument welche als erste angezeigt wird
	lastline=100                 % Letzte Zeile welche ins LaTeX Dokument übernommen wird
]{Quellcode/switch-after.java}

\section{2 Refactorings}

\subsubsection{Extract-Method}

\autoref{fig:extract-before} und \autoref{code:extract-method-before} zeigen die \textit{isValid}-Methode 
der \textit{CardDeck}-Klasse\footnote{Wie in \autoref{sec:smells}.}
\underline{vor} dem \textit{Extract-Method}-Refactoring und \autoref{fig:extract-after} und \autoref{code:extract-method-after} \underline{nach}
dem Refactoring durch \underline{Commit \texttt{cb55ad97}}. \\
Extract-Method wurde hier angewandt, um die Lesbarkeit der \textit{isValid}-Methode deutlich zu erhöhen, da für den 
ersten Teil nicht direkt klar war, was passiert, wodurch fast ein Code-Comment-Smell benötigt worden wäre, ohne Method-Extraction. 
Nun ist in der isValid-Methode klar, was im ersten Zeil (zählen der eigenen Karten) und im zweiten Teil (Vergleich mit der Soll-Konfiguration) 
passiert. Außerdem ist die neue extrahierte Methode \textit{countCardOccurrences} eine recht allgemeine Methode, die ggf. in Zukunft 
noch an anderer Stelle verwendet werden kann, um Code-Duplication zu verhindern. 

\lstinputlisting[
	label=code:extract-method-before,    % Label; genutzt für Referenzen auf dieses Code-Beispiel
	caption=\textit{isValid}-Methode \underline{vor} Method-Extraction (aus u.a. Commit \texttt{a0e99692}). Vgl. \autoref{code:switch-after}.,
	captionpos=b,               % Position, an der die Caption angezeigt wird t(op) oder b(ottom)
	style=EigenerJavaStyle,   % Eigener Style der vor dem Dokument festgelegt wurde
	firstline=35,                % Zeilennummer im Dokument welche als erste angezeigt wird
	lastline=50                 % Letzte Zeile welche ins LaTeX Dokument übernommen wird
]{Quellcode/switch-after.java}

\begin{figure}[H]
	\centering
	\includegraphics[width=0.85\textwidth]{Bilder/CardDeck_extract-before_structure.pdf} 
	\caption{UML-Diagramm von \textit{CardDeck} \underline{vor} der Method-Extraction.}
	\label{fig:extract-before}
\end{figure} 

\lstinputlisting[
	label=code:extract-method-after,    % Label; genutzt für Referenzen auf dieses Code-Beispiel
	caption=\underline{Nach} Method-Extraction der Methode \textit{countCardOccurrences} aus der \textit{isValid}-Methode in Commit \texttt{cb55ad97}.,
	captionpos=b,               % Position, an der die Caption angezeigt wird t(op) oder b(ottom)
	style=EigenerJavaStyle,   % Eigener Style der vor dem Dokument festgelegt wurde
	firstline=1,                % Zeilennummer im Dokument welche als erste angezeigt wird
	lastline=100                 % Letzte Zeile welche ins LaTeX Dokument übernommen wird
]{Quellcode/extract-method-after.java}

\begin{figure}[H]
	\centering
	\includegraphics[width=0.85\textwidth]{Bilder/CardDeck_extract-after_structure.pdf} 
	\caption{UML-Diagramm von \textit{CardDeck} \underline{nach} der Method-Extraction durch Commit \texttt{cb55ad97}. 
	Beachte die neue private Methode \textit{countCardOccurrences} in \textit{CardDeck}.}
	\label{fig:extract-after}
\end{figure} 



\subsubsection{Replace-Temp-with-Query (RTQ)}

\autoref{fig:rtq-before} und \autoref{code:rtq-before} zeigen die \textit{execute}-Methode 
der \textit{StartCommand}-Klasse
\underline{vor} dem \textit{Replace-Temp-with-Query-(RTQ)}-Refactoring und \autoref{fig:rtq-after} und \autoref{code:rtq-after} \underline{nach}
dem Refactoring durch \underline{Commit \texttt{827cd843}}. \\
Das Extrahieren der Berechnung des \textit{CardDeck} von der \textit{execute}-Methode in die \textit{deckFrom}-Methode 
hilft dabei, dass die execute-Methode das SRP einhält und nur noch zuständig ist, den Output der \textit{start}-Methode 
der \textit{Game}-Klasse an den User auszugeben. Außerdem wird dadurch der \textit{Long-Method}-Code-Smell der 
\textit{execute}-Methode beseitigt.

\lstinputlisting[
	label=code:rtq-before,    % Label; genutzt für Referenzen auf dieses Code-Beispiel
	caption=\textit{execute}-Methode \underline{vor} RTQ-Refactoring.,
	captionpos=b,               % Position, an der die Caption angezeigt wird t(op) oder b(ottom)
	style=EigenerJavaStyle,   % Eigener Style der vor dem Dokument festgelegt wurde
	firstline=1,                % Zeilennummer im Dokument welche als erste angezeigt wird
	lastline=100                 % Letzte Zeile welche ins LaTeX Dokument übernommen wird
]{Quellcode/query-before.java}

\begin{figure}[H]
	\centering
	\includegraphics[width=0.4\textwidth]{Bilder/StartCommand_before_structure.pdf} 
	\caption{UML-Diagramm von \textit{StartCommand} \underline{vor} dem RTQ-Refactoring.}
	\label{fig:rtq-before}
\end{figure} 

\lstinputlisting[
	label=code:rtq-after,    % Label; genutzt für Referenzen auf dieses Code-Beispiel
	caption=\textit{execute}-Methode und \textit{deckFrom}-Methode \underline{nach} RTQ-Refactoring in Commit \texttt{827cd843}.,
	captionpos=b,               % Position, an der die Caption angezeigt wird t(op) oder b(ottom)
	style=EigenerJavaStyle,   % Eigener Style der vor dem Dokument festgelegt wurde
	firstline=1,                % Zeilennummer im Dokument welche als erste angezeigt wird
	lastline=100                 % Letzte Zeile welche ins LaTeX Dokument übernommen wird
]{Quellcode/query-after.java}

\begin{figure}[H]
	\centering
	\includegraphics[width=0.4\textwidth]{Bilder/StartCommand_after_structure.pdf} 
	\caption{UML-Diagramm von \textit{StartCommand} \underline{nach} dem RTQ-Refactoring 
	durch Extraktion des Erstellens eines neuen Decks in Commit \texttt{827cd843}. 
	Beachte die neue private Methode \textit{countCardOccurrences} in \textit{CardDeck}.}
	\label{fig:rtq-after}
\end{figure} 
\chapter{Entwurfsmuster}

\section{Entwurfsmuster: Observer-Pattern}

\autoref{fig:observer} zeigt das Observer-Pattern, wobei ein \textit{GameEndObserver} von einem \textit{GameEndObservable} benachrichtigt 
wird, wenn das Spielende erreicht ist und mit einem \textit{GameResult} abgeschlossen wird. Das UML zeigt das 
konkrete GameEndObservable \textit{GameState}, welches seine GameEndObservers updated, wenn das GameResult über einen 
Setter geändert wird. Der konkrete GameEndObserver \textit{GameEndReporter} aus der Plugin-Schicht gibt dann 
das Spielergebnis an den User aus. Hierfür wird der GameEndReporter in der \textit{main}-Methode erzeugt und über 
DI an die höhere Schicht weitergegeben. \\ 
Das Observer-Pattern ist hier sehr sinnvoll, da das Spielende ein Zustand ist, 
der über viele verschiedene Wege und User-Befehle erreicht werden kann, wie zum Beispiel das Ziehen der letzten 
Karte, ohne dass noch eine Aktion möglich ist, oder der Spieler mit dem letzten möglichen Endeavor nicht erfolgreich ist usw. 
Daher ist es sinnvoll dies über ein Event abzuwickeln, damit nicht viele verschiedene Commands separat auf Spielende checken müssen, 
was auch gegen das SRP gehen würde. Außerdem erlaubt es eine sehr flexible Erweiterbarkeit, wenn zum Beispiel ein 
Highscore-Mechanismus eingebaut werden soll, der am Spielende die benötigten Züge abspeichert.      

\begin{figure}[H]
	\centering
	\includegraphics[width=0.6\textwidth]{Bilder/GameEndObserver_structure.pdf} 
	\caption{UML-Diagramm des Observer-Patterns zwischen \textit{GameState} und \textit{GameEndReporter}.}
	\label{fig:observer}
\end{figure} 

\section{Entwurfsmuster: Strategie}

\autoref{fig:strategy} zeigt das Stragie-Entwurfsmuster für die Logik der unterschiedlichen Endeavor-Möglichkeiten. 
Hierzu hält der Anwender \textit{RollHandler} eine Referenz auf die Strategie \textit{Rescue}, für die die 
Methode \textit{endeavor} implementiert werden muss. Die konkreten Strategien \textit{GuaranteedRescue} und \textit{PossibleRescue}
implementieren das Interface und können so flexibel von dem Anwender RollHandler verwendet werden. \\ 
Das Pattern ergibt hier sehr viel Sinn, da der RollHandler die konkrete Berechnung, ob das Endeavor geglückt ist oder nicht, 
nicht selbst durchführen muss, bzw. davon nichts wissen muss. Das unterstützt sehr stark das OCP, SRP und geringe Kopplung. 
Sollte eine weitere Form der Rescue eingeführt werden und möglich werden, ist das Programm so sehr flexibel und einfach 
erweiterbar.  

\begin{figure}[H]
	\centering
	\includegraphics[width=0.5\textwidth]{Bilder/Rescue_structure.pdf} 
    \caption{UML-Diagramm (gekürzt) des Strategie-Pattern der \textit{Rescue-Endeavor}-Logik. Zur besseren Übersichtlichkeit 
    wurden überflüssige Klassen ausgelassen.}
	\label{fig:strategy}
\end{figure} 

% ---- Literaturverzeichnis
\cleardoublepage
\renewcommand*{\chapterpagestyle}{plain}
\pagestyle{plain}
\pagenumbering{Roman}                   % Römische Seitenzahlen
\setcounter{page}{\numexpr\value{savepage}+1}
\printbibliography[title=Literaturverzeichnis]

% ---- Anhang
\appendix
\chapter{Anhang}

\section{Architekturen}

\pagebreak

\section{Quelltext-Implementation}

\lstinputlisting[
	label=code:quality,    % Label; genutzt für Referenzen auf dieses Code-Beispiel
	caption=Implementation des Quality-Maßes in Python zur Verwendung im Training und Testen von Keras-Modellen.,
	captionpos=b,               % Position, an der die Caption angezeigt wird t(op) oder b(ottom)
	style=EigenerPythonStyle,   % Eigener Style der vor dem Dokument festgelegt wurde
	firstline=1,                % Zeilennummer im Dokument welche als erste angezeigt wird
	lastline=50                 % Letzte Zeile welche ins LaTeX Dokument übernommen wird
]{Quellcode/quality.py}
%\clearpage
%\pagenumbering{Roman}  % römische Seitenzahlen für Anhang

\newpage
\end{document}
